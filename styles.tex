\usepackage[utf8]{inputenc}
\usepackage[T1]{fontenc}
\usepackage[a4paper,left=2cm,right=2cm,top=2cm,bottom=2cm,headheight=16pt]{geometry}
\usepackage{libertine}
\usepackage[pdftex]{graphicx}
\usepackage{totpages}
\usepackage[hidelinks]{hyperref}

%package pour faire des graphs
\usepackage{pgfplots}
\pgfplotsset{width=7cm,compat=1.8}
\pgfmathdeclarefunction{gauss}{2}{%
  \pgfmathparse{1/(#2*sqrt(2*pi))*exp(-((x-#1)^2)/(2*#2^2))}%
}
% \pgfmathdeclarefunction{ease}{}{%
%   \pgfmathparse{-(cos(pi*x)-1)/2}%
% }

\usepackage{helvet}
\renewcommand{\familydefault}{\sfdefault}
\usepackage[english, french]{babel}
\usepackage{subcaption}

\usepackage{setspace}
\onehalfspacing

% Change l'espacement entre deux paragraphes
\setlength{\parskip}{18pt}

% Liste à puces
\frenchbsetup{StandardLists=true}

% Édite style sous-titre figure et tableau
\usepackage[font=it]{caption}

% Change l'en-tête et le pied de page pour tout le document
\usepackage{fancyhdr}
\pagestyle{fancy}

\renewcommand{\chaptermark}[1]{\markboth{\thechapter.\ #1}{}}
\renewcommand{\sectionmark}[1]{\markright{\thesection.\ #1}}

\renewcommand{\headrulewidth}{0.5pt} 
\fancyhead[L]{\bfseries\leftmark}
\fancyhead[C]{}
\fancyhead[R]{\rightmark}

\renewcommand{\footrulewidth}{0.5pt}
\fancyfoot[L]{\textbf{Tugdual Le Pen}}
\fancyfoot[C]{}
\fancyfoot[R]{\thepage\ / \ref{TotPages}}

\fancypagestyle{plain}{ %
    \fancyhf{} % remove everything

    \renewcommand{\headrulewidth}{0pt} 
    \renewcommand{\footrulewidth}{0.5pt}
    \fancyfoot[L]{\textbf{Tugdual Le Pen}}
    \fancyfoot[C]{}
    \fancyfoot[R]{\thepage\ / \ref{TotPages}}
}

% Change le style des parties et sous partiess
\usepackage[explicit]{titlesec}
% change l'espacement au niveau du titre des chapitres
\titlespacing*{\chapter}
  {0pt}%  indent
  {0pt}% space before
  {12pt}% space after

\titlespacing*{\section}
  {0.6cm}%  indent
  {0pt}% space before
  {0pt}% space after

\titlespacing*{\subsection}
  {1.2cm}%  indent
  {0pt}% space before
  {0pt}% space after

% Définie les couleurs utilisées dans le rapport
\usepackage{color}

\definecolor{darkblue}{rgb}{0.11, 0.30, 0.66}
\definecolor{lightblue}{rgb}{0.18, 0.42, 0.86}
\definecolor{gray}{rgb}{0.4,0.4,0.4}
\definecolor{black}{rgb}{0.0,0.0,0.0}
\definecolor{green}{rgb}{0.4, 0.8, 0.2}

\definecolor{purple_bcom}{RGB}{190, 100, 255}
\definecolor{blue_bcom}{RGB}{36, 210, 246}
\definecolor{green_bcom}{RGB}{0, 205, 115}
\definecolor{red_bcom}{RGB}{255, 79, 79}
\definecolor{orange_bcom}{RGB}{255, 180, 0}


%% Style Chapitre
% -------------------------------------------------------
% Avec numéro
\titleformat{\chapter}[hang] 
    {\fontsize{24pt}{0pt}\selectfont \bfseries}
    {\textcolor{red_bcom} 
    {\thechapter. <#1>}}
    {0pt}
    {\huge}

% Sans numéro
\titleformat{name=\chapter,numberless}[hang] 
    {\fontsize{24pt}{0pt}\selectfont \bfseries}
    {\textcolor{red_bcom} 
    {<#1>}}
    {0pt}
    {\huge}
% -------------------------------------------------------

%% Style Section
% -------------------------------------------------------
% Avec numéroté
\titleformat{\section}[hang]
    {\fontsize{16pt}{0pt}\selectfont \bfseries}
    {\textcolor{orange_bcom} 
    {\thesection.\ \{#1\}}}
    {10pt}
    {\Large}

% Sans numéro
\titleformat{name=\section,numberless}[display] 
    {\fontsize{16pt}{0pt}\selectfont \bfseries}
    {\textcolor{orange_bcom} 
    {\thechapter.\thesection \{#1\}}}
    {10pt}
    {\Large}
% -------------------------------------------------------

%% Style Subsection
% -------------------------------------------------------
% Avec numéroté
\titleformat{\subsection}[hang]
    {\fontsize{16pt}{0pt}\selectfont \bfseries}
    {\textcolor{black} 
    {\thesubsection.\ [#1]}}
    {10pt}
    {\Large}

% Sans numéro
\titleformat{name=\subsection,numberless}[display] 
    {\fontsize{16pt}{0pt}\selectfont \bfseries}
    {\textcolor{black} 
    {\thesubsection.\ [#1]}}
    {10pt}
    {\Large}
% -------------------------------------------------------

% Numérotation des chapitre en chiffre Romain
\renewcommand{\thechapter}{\Roman{chapter}}

% Créer commande pour lien url
\newcommand{\link}[1]{{\color{green_bcom}\href{#1}{#1}}}
\newcommand{\linkrename}[2]{{\color{green_bcom}\href{#2}{#1}}} % #1 = rename ; #2 = link

% Éviter les orphelins en début ou fin de page
\widowpenalty=10000 
\clubpenalty=10000 
\raggedbottom

% Nouvelle commande pour ajouter des commentaires sur le rapport
\newcommand{\comment}[1]{\emph{\color{purple_bcom} \% #1}}

% Commande emoji coeur
\newcommand{\heart}{\ensuremath\heartsuit}

% Commande pour ajouter une image
\newcommand{\addimage}[3][0.7]{
  \begin{figure}[ht]
    \centering
    \includegraphics[width=#1\textwidth]
                    {datas/#2}
    \caption{#3}
    \label{#2}
\end{figure}
} % #1 = Image size (default = 0.7) ; #2 = Image name & label ; #3 = Caption