\addcontentsline{toc}{chapter}{Résumé}
\chapter*{Résumé}

\par
Pour valider ma 5ème et dernière année de mon cycle ingénieur en Technologie de l’Information avec spécialité Imagerie Numérique, j’ai effectué un stage d’une durée de six mois dans l’Institut de Recherche b<>com. C'est un institut qui fournit son expertise et ses technologies en numérique aux entreprises de différents secteurs (santé, défense, industrie 4.0, etc.).

\par
J'ai rejoint plus précisement l'équipe IMT (Technologies Immersives et Médicales) qui est sépecialisée dans la réalité virtuelle/augmentée pour le domaine de la santé et de l'industrie 4.0. L'un des projets de cette équipe est le framework SolAR, un support rempli d'outils utiles pour développer des applications en réalité virtuelle ou augmentée.

\par
Mon objectif est d'intégrer un programme capable de créer un environnement 3D à partir d'un lot d'image en couleur au sein du framework SolAR.

\vspace{40pt}

\selectlanguage{english}
\color{gray}

\par
Pour valider ma 5ème et dernière année de mon cycle ingénieur en Technologie de l’Information avec spécialité Imagerie Numérique, j’ai effectué un stage d’une durée de six mois dans l’Institut de Recherche b<>com. C'est un institut qui fournit son expertise et ses technologies aux entreprises de différents secteurs (santé, défense, industrie 4.0, etc.).

\par
J'ai rejoint plus précisement l'équipe IMT (Technologies Immersives et Médicales) qui est sépecialisée dans la réalité virtuelle/augmentée pour le domaine de la santé et de l'industrie 4.0. L'un des projets de cette équipe est le framework SolAR, un support rempli d'outils utiles pour développer des applications en réalité virtuelle ou augentée.

\par
Mon objectif est d'intégrer un programme capable de créer un environnement 3D à partir d'un lot d'image en couleur au sein du framework SolAR.

\selectlanguage{french}
\color{black}