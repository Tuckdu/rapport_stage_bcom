\chapter{Présentation de b<>com}

\par
Depuis sa création en 2012, l'Institut de Recherche Technologie b<>com a pour but de ressourcer les talents et expertises afin d’être un fournisseur de technologies pour les entreprises souhaitant accélérer leur compétitivité grâce au numérique. b<>com est basé sur un modèle de co-investissement unique qui génère technologies, connaissances et savoir-faire.

\addimage{batiment_bcom.jpg}{Site b<>com de Rennes}

\par
Les technologies développées dans cet institut sont conçues pour les infrastructures numériques, les industries culturelles et créatives, la santé, la défense, la sécurité et l’industrie 4.0. L'institut représente des grands groupes indutriels (Orange, Harmonic, Nokia, Mitsubishi Electric, Airbus), des organismes de santé (CHU de Rennes, CHU de Brest), des partenaires académiques (INRIA, INSA, INSERM, Université Rennes 1, institut Mines Télécom Atlantique) et un ensemble de PME bretonnes.

\par
b<>com s'implique dans des projets à échelle européenne comme le \linkrename{5G tour}{https://b-com.com/nous-connaitre/galaxie-bcom/5g-tours} ou \linkrename{ARTwin}{https://b-com.com/nous-connaitre/galaxie-bcom/artwin}, mais aussi dans des organes de standardisation mondiaux et alliances professionnelles (voir \ref{standardisation_alliance.png}).

\addimage[0.7]{standardisation_alliance.png}{Organes de standardisation et alliances}

\comment{nos chercheurs et ingénieurs évoluent sur le campus principal de Rennes et nos sites de Paris, Brest et Lannion.L’entreprise regroupe en 2020 plus de 300 collaborateurs (chiffres a mettre a jour).}

\comment{différents labos : \textbf{Technologies Immersives et Médicales}, Confiance et Sécurité, Nouveaux Contenus Média, Technologies Facteurs Humains, Connectivité Avancée, Intelligence Artificielle}