\chapter{Présentation de b<>com}

\par
Depuis sa création en 2012, l'Institut de Recherche Technologie b<>com a pour but de ressourcer les talents et expertises afin d’être un fournisseur de technologies pour les entreprises souhaitant accélérer leur compétitivité grâce au numérique. b<>com est basé sur un modèle de co-investissement unique qui génère technologies, connaissances et savoir-faire.

\addimage[0.5]{batiment_bcom.jpg}{Site b<>com de Rennes}

\par
Les technologies développées dans cet institut sont conçues pour les infrastructures numériques, les industries culturelles et créatives, la santé, la défense, la sécurité et l’industrie 4.0. L'institut représente des grands groupes indutriels (Orange, Harmonic, Nokia, Mitsubishi Electric, Airbus), des organismes de santé (CHU de Rennes, CHU de Brest), des partenaires académiques (INRIA, INSA, INSERM, Université Rennes 1, institut Mines Télécom Atlantique) et un ensemble de PME bretonnes.

\par
b<>com s'implique dans des projets à échelle européenne comme le \linkrename{5G tour}{https://b-com.com/nous-connaitre/galaxie-bcom/5g-tours} ou \linkrename{ARTwin}{https://b-com.com/nous-connaitre/galaxie-bcom/artwin}, mais aussi dans des organes de standardisation mondiaux et alliances professionnelles (voir image \ref{standardisation_alliance.png}).

\addimage{standardisation_alliance.png}{Organes de standardisation et alliances}

\par
Les becomiens (employés de b<>com) évoluent sur le campus principal de Rennes, où j'ai effectué mon stage de 6 mois, et les sites de Paris, Brest et Lannion. L’entreprise regroupe en 2021 plus de 300 collaborateurs.

\newpage

\par
L'institut est divisé en plusieurs laboratoires qui ont chacun leur propre spécialité et leurs propres technologies. Il y en a six au total :
\begin{itemize}
    \item Le laboratoire \textbf{\linkrename{Technologies Immersives et Médicales}{https://b-com.com/nous-connaitre/nos-labos/technologies-immersives-et-medicales}} (IMT)
    \item Le laboratoire \linkrename{Confiance et Sécurité}{https://b-com.com/nous-connaitre/nos-labos/confiance-et-securite} (TS)
    \item Le laboratoire \linkrename{Nouveaux Contenus Média}{https://b-com.com/nous-connaitre/nos-labos/nouveaux-contenus-media} (AMC)
    \item Le laboratoire \linkrename{Technologies Facteurs Humains}{https://b-com.com/nous-connaitre/nos-labos/technologies-facteurs-humains} (HFT)
    \item Le laboratoire \linkrename{Connectivité Avancée}{https://b-com.com/nous-connaitre/nos-labos/connectivite-avancee} (AC)
    \item Le laboratoire \linkrename{Intelligence Artificielle}{https://b-com.com/nous-connaitre/nos-labos/intelligence-artificielle} (AI)
\end{itemize}

\par
C'est dans le laboratoire IMT que s'est déroulé mon stage. Ce laboratoire est spécialisé dans la vision par ordinateur, l'estimation de pose et la visualisation 3D. Il travaille sur des projets qui repose principalement sur la réalité virtuelle ou augmentée pour le domaine de la santé et de l'industrie 4.0. C'est au sein de ce laboratoire que j'ai intégré l'équipe qui travaille sur le projet SolAR.
